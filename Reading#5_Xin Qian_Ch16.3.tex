\documentclass[12pt]{article}

\usepackage{fancyhdr} % Required for custom headers
\usepackage{lastpage} % Required to determine the last page for the footer
\usepackage{extramarks} % Required for headers and footers
\usepackage{graphicx} % Required to insert images
\usepackage{lipsum} % Used for inserting dummy 'Lorem ipsum' text into the template

% Margins
\topmargin=-0.25in
\evensidemargin=0in
\oddsidemargin=0in
\textwidth=6.5in
\textheight=9.0in
\headsep=0.15in 

\linespread{1.2} % Line spacing

% Set up the header and footer
%\pagestyle{fancy}
\setlength\parindent{0pt} % Removes all indentation from paragraphs

%----------------------------------------------------------------------------------------
%	DOCUMENT STRUCTURE COMMANDS
%	Skip this unless you know what you're doing
%----------------------------------------------------------------------------------------

% Header and footer for when a page split occurs within a problem environment
\newcommand{\enterProblemHeader}[1]{
\nobreak\extramarks{#1}{#1 continued on next page\ldots}\nobreak
\nobreak\extramarks{#1 (continued)}{#1 continued on next page\ldots}\nobreak
}

% Header and footer for when a page split occurs between problem environments
\newcommand{\exitProblemHeader}[1]{
\nobreak\extramarks{#1 (continued)}{#1 continued on next page\ldots}\nobreak
\nobreak\extramarks{#1}{}\nobreak
}

\setcounter{secnumdepth}{0} % Removes default section numbers
\newcounter{homeworkProblemCounter} % Creates a counter to keep track of the number of problems

\newcommand{\homeworkProblemName}{}
\newenvironment{homeworkProblem}[1][Problem \arabic{homeworkProblemCounter}]{ % Makes a new environment called homeworkProblem which takes 1 argument (custom name) but the default is "Problem #"
\stepcounter{homeworkProblemCounter} % Increase counter for number of problems
\renewcommand{\homeworkProblemName}{#1} % Assign \homeworkProblemName the name of the problem
\section{\homeworkProblemName} % Make a section in the document with the custom problem count
\enterProblemHeader{\homeworkProblemName} % Header and footer within the environment
}{
\exitProblemHeader{\homeworkProblemName} % Header and footer after the environment
}

\newcommand{\problemAnswer}[1]{ % Defines the problem answer command with the content as the only argument
\noindent\framebox[\columnwidth][c]{\begin{minipage}{0.98\columnwidth}#1\end{minipage}} % Makes the box around the problem answer and puts the content inside
}

\newcommand{\homeworkSectionName}{}
\newenvironment{homeworkSection}[1]{ % New environment for sections within homework problems, takes 1 argument - the name of the section
\renewcommand{\homeworkSectionName}{#1} % Assign \homeworkSectionName to the name of the section from the environment argument
\subsection{\homeworkSectionName} % Make a subsection with the custom name of the subsection
\enterProblemHeader{\homeworkProblemName\ [\homeworkSectionName]} % Header and footer within the environment
}{
\enterProblemHeader{\homeworkProblemName} % Header and footer after the environment
}
   
%----------------------------------------------------------------------------------------
%	NAME AND CLASS SECTION
%----------------------------------------------------------------------------------------

\newcommand{\hmwkTitle}{Reading Summary Ch 8-8.5} 
\newcommand{\hmwkClass}{11642}
\newcommand{\hmwkAuthorName}{Xin Qian} % Your name

%----------------------------------------------------------------------------------------
%	TITLE PAGE
%----------------------------------------------------------------------------------------

\title{
\textmd{\textbf{\hmwkClass:\ \hmwkTitle}
}}

\author{\textbf{\hmwkAuthorName}}
 % Insert date here if you want it to appear below your name

%----------------------------------------------------------------------------------------

\begin{document}
\subsection{11741 Reading Summary, Ch 16.3 \\Xin Qian (xinq@cs.cmu.edu)}
Clustering has a goal of getting the clusters highly cohesive and inter-cluster dissimilar, which defines the quality of a clustering and yields an internal criterion. As an alternative, there's the direct evaluation from the information retrieval application. This section infroduces 4 external criteria to measure clustering quality. We resort to gold standard or evaluation benchmark in computing an external criterion. \\

Purity is calculated as the number of correct assignment divided by the total number of documents. Purity has a value from 0 to 1. It is easy to have a high purity when the number of clusters is large but we cannot increase the number of clusters without a limit at the cost of the clustering quality. \\

Normalized mutual information is the amount of class information acquired given the cluster divided by the arithmetic mean of entropy. A minimum value of mutual information is 0, which indicates we still cannot tell which cluster a document belongs to even when we know its class. Maximum mutual information happens when a clustering perfectly replays the original classes or further subdivides the class hierarchy. However, MI has zero penalty on large clustering cardinality. We thus normalize MI by the entropy as entropy increases with the number of cluster. NMI has a value between 0 and 1. \\

Random index measures the percentage of correct pairwise decisions. TP is a correct assignment for two similar documents. TN is a correct assignment to separate two dissimilar documents. FN is a wrong assignment to separate two similar documents. FP is a wrong assignment for two dissimilar documents to be put together. RI is calculated as (TP+TN)/(TP+FP+FN+TN). This assumes separating similar documents is of equal harm as putting paris of dissimilar documents in the same cluster. Random index is derived from the perspective of information theory. The process of clustering can be viewed as a series of dichotomy decision. \\

F measure revisits this formula by plugging in a boost factor $\beta >$ 1. Due to its flexibility, it is advantageous to evaluate clustering with F. 








%----------------------------------------------------------------------------------------

\end{document}
