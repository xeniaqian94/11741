\documentclass[10pt]{article}

\usepackage{fancyhdr} % Required for custom headers
\usepackage{lastpage} % Required to determine the last page for the footer
\usepackage{extramarks} % Required for headers and footers
\usepackage{graphicx} % Required to insert images
\usepackage{lipsum} % Used for inserting dummy 'Lorem ipsum' text into the template

% Margins
\topmargin=-0.25in
\evensidemargin=0in
\oddsidemargin=0in
\textwidth=6.5in
\textheight=9.0in
\headsep=0.15in 

\linespread{1.2} % Line spacing

% Set up the header and footer
%\pagestyle{fancy}
\setlength\parindent{0pt} % Removes all indentation from paragraphs

%----------------------------------------------------------------------------------------
%	DOCUMENT STRUCTURE COMMANDS
%	Skip this unless you know what you're doing
%----------------------------------------------------------------------------------------

% Header and footer for when a page split occurs within a problem environment
\newcommand{\enterProblemHeader}[1]{
\nobreak\extramarks{#1}{#1 continued on next page\ldots}\nobreak
\nobreak\extramarks{#1 (continued)}{#1 continued on next page\ldots}\nobreak
}

% Header and footer for when a page split occurs between problem environments
\newcommand{\exitProblemHeader}[1]{
\nobreak\extramarks{#1 (continued)}{#1 continued on next page\ldots}\nobreak
\nobreak\extramarks{#1}{}\nobreak
}

\setcounter{secnumdepth}{0} % Removes default section numbers
\newcounter{homeworkProblemCounter} % Creates a counter to keep track of the number of problems

\newcommand{\homeworkProblemName}{}
\newenvironment{homeworkProblem}[1][Problem \arabic{homeworkProblemCounter}]{ % Makes a new environment called homeworkProblem which takes 1 argument (custom name) but the default is "Problem #"
\stepcounter{homeworkProblemCounter} % Increase counter for number of problems
\renewcommand{\homeworkProblemName}{#1} % Assign \homeworkProblemName the name of the problem
\section{\homeworkProblemName} % Make a section in the document with the custom problem count
\enterProblemHeader{\homeworkProblemName} % Header and footer within the environment
}{
\exitProblemHeader{\homeworkProblemName} % Header and footer after the environment
}

\newcommand{\problemAnswer}[1]{ % Defines the problem answer command with the content as the only argument
\noindent\framebox[\columnwidth][c]{\begin{minipage}{0.98\columnwidth}#1\end{minipage}} % Makes the box around the problem answer and puts the content inside
}

\newcommand{\homeworkSectionName}{}
\newenvironment{homeworkSection}[1]{ % New environment for sections within homework problems, takes 1 argument - the name of the section
\renewcommand{\homeworkSectionName}{#1} % Assign \homeworkSectionName to the name of the section from the environment argument
\subsection{\homeworkSectionName} % Make a subsection with the custom name of the subsection
\enterProblemHeader{\homeworkProblemName\ [\homeworkSectionName]} % Header and footer within the environment
}{
\enterProblemHeader{\homeworkProblemName} % Header and footer after the environment
}
   
%----------------------------------------------------------------------------------------
%	NAME AND CLASS SECTION
%----------------------------------------------------------------------------------------

\newcommand{\hmwkTitle}{Reading Summary Ch 18} 
\newcommand{\hmwkClass}{11642}
\newcommand{\hmwkAuthorName}{Xin Qian} % Your name

%----------------------------------------------------------------------------------------
%	TITLE PAGE
%----------------------------------------------------------------------------------------

\title{
\textmd{\textbf{\hmwkClass:\ \hmwkTitle}
}}

\author{\textbf{\hmwkAuthorName}}
 % Insert date here if you want it to appear below your name

%----------------------------------------------------------------------------------------

\begin{document}
\subsection{11741 Reading Summary, Ch 18 \\Xin Qian (xinq@cs.cmu.edu)}
\subsection{Linear algebra review}
The rank of a matrix is the minumum number of linearly independent rows or columns. We could get the eigenvalues of a matrix by solving the characteristic equation. The effect of multiplication for an arbitrary vector by S is determined by the eigenvalues and eigenvectors of S. This section ends with the intuition interpretation on the effect of small eigenvalues and their eigenvectors on a matrix-vector product is small. A square matrix can be factored into the product of matrices derived from its eigenvectors. The low-rank approximations to term-document matrics are built on the symmetric diagonal decomposition theorem.
\subsection{Term-document matrics}
Term-document matrix C are likely to have M != N. And C is very likely to be asymmetric. Performing matrix decomposition on C multiplied by C transpose (a square symmetric matrix) gives the singular value decomposition, which is an extension of the symmetric diagonal decomposition. 
\subsection{Low-rank approximations}
This section states a matrix approximation problem where we want to find a M by N matrix of rank at most k to minimize the Frobenius norm (discrepancy) of the matrix difference. Replacing small eigenvalues by zero creates the best rank-k approximation to C. This section ends with discussing why truncating the smallest r-k singular values helps generate low error rank-k approximation. 
\subsection{Latent semantic indexing}
LSI sometimes is refered to LSA as well. Casting queries into the low-rank representation and computing corresponding query-document similarity scores is known as latent semantic indexing. Recall that the vector space representation sufferes from the inexpressiveness of synonymy and polysemy as well as the large size of the term-document matrix. We might want to use the SVD to construct a low-rank approcimation matrix to the original term-document matrix. A popular choice of k is usually in the low hundreds. LSI for query-document similarity matches was as follows, we map a query in the LSI space by the SVD transformation on the term-document matrix. This process is not specific to only query vector, but can be a document vector who wishes to be added in the collection. We are expecting to retain the retrieval quality from dimension reduction and improve such quality by grouping together similar co-occurrence terms. LSI has two major drawbacks of vector space retrieval: unable to express negations and unable to enforce Boolean conditions. Experiment results expects and verified that reducing k tends to increase recall. LSI can be interpreted as a soft clustering and each document has a fractional membership in each dimension. 



%----------------------------------------------------------------------------------------

\end{document}
