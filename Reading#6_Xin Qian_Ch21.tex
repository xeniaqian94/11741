\documentclass[12pt]{article}

\usepackage{fancyhdr} % Required for custom headers
\usepackage{lastpage} % Required to determine the last page for the footer
\usepackage{extramarks} % Required for headers and footers
\usepackage{graphicx} % Required to insert images
\usepackage{lipsum} % Used for inserting dummy 'Lorem ipsum' text into the template

% Margins
\topmargin=-0.25in
\evensidemargin=0in
\oddsidemargin=0in
\textwidth=6.5in
\textheight=9.0in
\headsep=0.15in 

\linespread{1.2} % Line spacing

% Set up the header and footer
%\pagestyle{fancy}
\setlength\parindent{0pt} % Removes all indentation from paragraphs

%----------------------------------------------------------------------------------------
%	DOCUMENT STRUCTURE COMMANDS
%	Skip this unless you know what you're doing
%----------------------------------------------------------------------------------------

% Header and footer for when a page split occurs within a problem environment
\newcommand{\enterProblemHeader}[1]{
\nobreak\extramarks{#1}{#1 continued on next page\ldots}\nobreak
\nobreak\extramarks{#1 (continued)}{#1 continued on next page\ldots}\nobreak
}

% Header and footer for when a page split occurs between problem environments
\newcommand{\exitProblemHeader}[1]{
\nobreak\extramarks{#1 (continued)}{#1 continued on next page\ldots}\nobreak
\nobreak\extramarks{#1}{}\nobreak
}

\setcounter{secnumdepth}{0} % Removes default section numbers
\newcounter{homeworkProblemCounter} % Creates a counter to keep track of the number of problems

\newcommand{\homeworkProblemName}{}
\newenvironment{homeworkProblem}[1][Problem \arabic{homeworkProblemCounter}]{ % Makes a new environment called homeworkProblem which takes 1 argument (custom name) but the default is "Problem #"
\stepcounter{homeworkProblemCounter} % Increase counter for number of problems
\renewcommand{\homeworkProblemName}{#1} % Assign \homeworkProblemName the name of the problem
\section{\homeworkProblemName} % Make a section in the document with the custom problem count
\enterProblemHeader{\homeworkProblemName} % Header and footer within the environment
}{
\exitProblemHeader{\homeworkProblemName} % Header and footer after the environment
}

\newcommand{\problemAnswer}[1]{ % Defines the problem answer command with the content as the only argument
\noindent\framebox[\columnwidth][c]{\begin{minipage}{0.98\columnwidth}#1\end{minipage}} % Makes the box around the problem answer and puts the content inside
}

\newcommand{\homeworkSectionName}{}
\newenvironment{homeworkSection}[1]{ % New environment for sections within homework problems, takes 1 argument - the name of the section
\renewcommand{\homeworkSectionName}{#1} % Assign \homeworkSectionName to the name of the section from the environment argument
\subsection{\homeworkSectionName} % Make a subsection with the custom name of the subsection
\enterProblemHeader{\homeworkProblemName\ [\homeworkSectionName]} % Header and footer within the environment
}{
\enterProblemHeader{\homeworkProblemName} % Header and footer after the environment
}
   
%----------------------------------------------------------------------------------------
%	NAME AND CLASS SECTION
%----------------------------------------------------------------------------------------

\newcommand{\hmwkTitle}{Reading Summary Ch 21} 
\newcommand{\hmwkClass}{11642}
\newcommand{\hmwkAuthorName}{Xin Qian} % Your name

%----------------------------------------------------------------------------------------
%	TITLE PAGE
%----------------------------------------------------------------------------------------

\title{
\textmd{\textbf{\hmwkClass:\ \hmwkTitle}
}}

\author{\textbf{\hmwkAuthorName}}
 % Insert date here if you want it to appear below your name

%----------------------------------------------------------------------------------------

\begin{document}
\subsection{11741 Reading Summary, Ch 21 \\Xin Qian (xinq@cs.cmu.edu)}
Web search places great emphasis on link analysis. Link analysis for web search is the intellectual decedent of the field of citation analysis. A simple measurement of a web page's quality by the number of in-links is not robust to link spam phenomenon since people might set up several web pages pointing to a target page to boost its in-link count. Link analysis is useful in crawl suggestion. \\
Treating the web as a graph underlies two intuitions: first, the anchor text is a good page descriptor; second, a hyperlink from A to B is an endorsement by author of page A to page B. Anchor text bridges the gap between the original content/terms in a web page and the external description of the page. Current web search engines assign considerable weights to anchor text terms. However, there's sometimes orchestrated anchor text as a form of spam. Extended anchor text is often useful as the original anchor text, depending on the window width. \\
PageRank give every node in the web graph a numerical score from 0 to 1. The PageRank score can be combined as a feature into web search. PageRank is modelled as a random walker who visits some nodes more often than other, indicating these are important pages. Teleport operation solves the problem when a node has no out-links. A Markov chain consists of N states, corresponding to each web page. N by N transition probability matrix has each entry as a transition probability that depends only on the current state. Ergodic Markov chain means for all pairs of state, starting from state i at time 0, after a certain time $T_0$, the probability of being in the state j is greater than 0. The random walk along with teleporting generates a unique distribution of steady-state probabilities over the states. PageRank is a static measure of web page quality. Topic-specific PageRank teleports to a random web page non-uniformly, only a subset of web pages over which the random walk has a steady-state distribution. Personalized PageRank assumes that individual interest can be modelled as a linear combination of topic page distributions. Personalized PageRank vector can be constructed linearly from its underlying topic-specific PageRanks. \\
Dividing web pages into hub pages and authority pages, we might want to use the hub pages to discover the authority pages. We can iteratively compute the hub score and the authority score for every web page in the subset of the web containing good hub and authority pages. This is called HITS, the Hyperlink-Induced Topic Search. Sometimes good authority pages might not contain a specific query term. We include the root set of pages and the base set of pages to compile an adequate subset of the Web. Cross-language retrieval phenomenon are observed. Examples on the query japan elementary schools shows this phenomenon.




%----------------------------------------------------------------------------------------

\end{document}
